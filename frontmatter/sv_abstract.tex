% !TeX root = ../main.tex
% !TeX spellcheck = en_SV
%: Abstract swedish


\begin{abstract}
%\textbf{Bakgrund:}
Melanom är en form av hudcancer som tenderar att vara dödlig. Förekomsten av melanom är för närvarande på den högsta nivån som någonsin registrerats i Europa, Nordamerika och Oceanien. Chansen för överlevnad kan ökas avsevärt om hudskadorna identifieras i dermatoskopiska bilder i ett tidigt skede, men klassificering av hudskador är otroligt utmanande. Med metoder för djupinlärning har klassificering av hudsjukdomar i vissa fall gett bättre resultat än hudläkares diagnoser, vilket ger större möjligheter att rädda liv.

%\textbf{Syfte:}
Denna avhandling presenterar en genomgång av våra forskningsartiklar om klassificering av hudskador med hjälp av djupinlärning. När det gäller vår forskning har jag fyra mål som handlar om forskningens frontlinjearbete, små datamängder, obalans i data och om att förbättra noggrannheten. I detta avhandlingsarbete diskuterar jag hur djupinlärning kan klassificera hudsjukdomar, sammanfattar de problem som kvarstår i detta skede och diskuterar utsikterna för framtiden.

%\textbf{Metoder:}
För ovanstående mål studerade och sammanfattade jag först mer än 200 högkvalitativa artiklar publicerade under fem år. Jag använde sedan tre versioner av You only look once (Yolo) för att upptäcka hudskador. Även om det bara fanns 200 bilder var testet mycket effektivt för upptäckt. Jag tillämpade en femdelad algoritm på Vgg-16, tränade fem modeller och sammanfogade dem för att lösa problemet med små datamängder. För att förbättra noggrannheten försökte jag också kombinera en sju-punkts checklista, förstärkt med maskininlärning, med tre olika grundstommar. Eftersom inlärningshastigheten starkt påverkar modellträningen använde jag cosinus-inlärningshastigheten. Sedan försökte jag också använda hybridmodellen, som kombinerade konvolutionella neurala nätverk (CNN) och transformator för att träna datasetet, och tillämpade fokalförlust för att balansera den extremt obalanserade vikten av datan.

%\textbf{Slutsatser:}
Förutom att högkvalitativa datamängder och högpresterande datorer är extremt viktiga i forskningen och tillämpningen av djupinlärning, kan optimeringen av maskininlärningsalgoritmer för hudskador vara oändlig.


\end{abstract}